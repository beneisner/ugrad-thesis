\section{Problem Background and Related Work}

\TODO{Flesh out more background on CNNs}

The problem of determining the connectivity of a brain falls in the sub-field of connectomics, which has been a lively area for research for over 30 years. The first full connectome of an organism was created in 1986, producing the mapping of the brain of \textit{C. elegans} \cite{White1986}. Since then, partial and full topological and connectivity maps have been created on various organism, often using electron microscopy and careful hand-reconstruction to do so. More recently, by genetically modifying organisms to produce proteins that become phosphorescent in the presence of calcium (calcium is released across the synapse between a dendrite and an axon when a neuron is fires), researchers have been able to monitor both brain activity and neural structure using video photography in the visible spectrum \cite{Nguyen2015}.

In terms of the computational approaches to automating connectome reconstruction, many groups have attempted to create systems to accomplish near-human accuracy when labelling neurons in the brain. In the early 2010's, a group of researchers published an open dataset and created a global challenge to use machine learning methods to label neurons in image slices of a brain, resulting in the creation of models with near-human accuracy \cite{Arganda-Carreras2015}. More recently, researchers at Princeton University applied similar approaches to automatically detecting neurons in videos formed with the calcium imaging techniques mentioned above \cite{Apthorpe2016}.

The work in the field continues to improve mapping techniques, but our understanding of learning methods in both computational neuroscience and computer vision are in their infancy, and are not yet practical for deployment on a larger organism (i.e. a human) due to computational and accuracy issues. This project hopes to improve on existing methods, even if incrementally, so that we can better understand what an efficient mapping approach capable of mapping a full brain might look like.